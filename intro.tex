\section{Introduction}
\label{sec:intro}

We will begin by describing the theorem that we intend to prove then
we will get into some motivation for it's proof. Our theorem will
``classify'' surfaces, or \emph{2-manifolds} by showing that each
\emph{compact surface} is \emph{equivalent} to one in a countable family of
canonical surfaces. The meaning of this might not be clear at
first as we haven't yet properly defined what we mean by
\emph{surface}, \emph{compact} or \emph{equivalent}; a proper
definition of these terms will be given after the next section where we
formally introduce these concepts and can then make a proper statement
of the theorem. 

Even so, the theorem can be quite illustratively described in a
qualitative manner. And we can also provide an overview and high-level
motivation for the proof. The notion of \emph{surface} is quite
familiar to us in common parlance, these include planes and the
``outermost layer'' of most of everyday objects we interact with,
though we need to first introduce topological spaces to give us a
formal definition. Our preliminary section will be dedicated to
generalising many concepts from metric spaces via topological spaces.

We have been talking about \emph{equivalence} of surfaces in the
preceding paragraphs. To be more precise, we really mean
\emph{homeomorphic}, that is there exists an invertible function
between the two surfaces and both it and its inverse are
continuous. For this we will develop notions of continuity in the next
section, which will be more general than what we have already
encountered in metric spaces. \emph{Compactness} also has a topological
definition that is distinct and not in general equivalent to sequential
compactness we are familiar with from metric spaces.

\subsection{Motivation for the Proof}
\label{sec:intro:motivation}

We complete our preliminary section by introducing a particular
topological space, \emph{the quotient topology}. This will provide us
with a toolkit to deform surfaces without tearing including most usefully, the
ability to \emph{identify edges}, or  `cut and paste' patches of the
surface, being certain that surfaces before and after this action are
homeomorphic. To actually prove the classification theorem, having introduced our toolkit, we follow the proof given by Massey \cite{mass}.

An important assumption we make (and exclude from the proof) is that
compact surfaces can be \emph{triangulated} i.e.\ cut up into finitely
many triangles. We will properly define triangulation later, but to
see a proof that compact surfaces can be triangulated. Having ``cut up'' our surface into finitely many triangles, we will
paste them onto the plane $\mathbb{R}^2$ and arrange them such that a
polygon is formed. From here we will \emph{identify edges} and
\emph{identify vertices} of the polygon till we have a polygon that
matches one of our countably many prototypes.




%%% Local Variables:
%%% mode: latex
%%% TeX-master: "main"
%%% End:
