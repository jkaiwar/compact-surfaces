%% Tomorrow
%% surfaces example
%% extend to n manifold
%% orientability intuition
%% research definition of orientablity
%% prototype drawings and quotient space formations
%% connected sum definitions and n-torus quotient space formation
%% lemma 7.1


\section{Surfaces and their Classification}
\label{sec:surf}

We will finally be able to give a proper topological definition for
surfaces, and soon we will be able to state the theorem.

\subsection{Surfaces and Orientability}
\label{sec:surf:surfaces}

We begin by recalling the metric topology from example
\ref{exmp:metric}. With this topology in mind for the metric space
$\mathbb{R}^2$ under the euclidean metric, we have the following
definition

\begin{defn}
  We say that $S$ is a \emph{surface} if it is a Hausdorff space for
  which every point has a neighbourhood that is homeomorphic to the
  open unit ball $\{\mathbf{x} \in \mathbb{R}^2 \mid \lvert \mathbf{x}
  \rvert < 1 \}$.
\end{defn}

We present the following examples.

\begin{exmp}
  The plane $\mathbb{R}^2$ is trivially a surface; each point is
  contained in $\mathbb{R}^2$, an open set that is homeomorphic to the
  unit ball $B_1(0)$ via $f:B_1(0) \rightarrow \mathbb{R}^2$, given by
  \[
    x \mapsto \frac{x}{x^2 - 1}.
  \]

  It is easy to see as a simple extension that any open subset of
  $\mathbb{R}^2$ is a surface.
\end{exmp}


We will see many more examples of surfaces in section
\ref{sec:surf:prototype} but we will present one more example of a
surface to introduce the concept of \emph{orientability}. Here our
example will be defined as a \emph{quotient space}. The reader is
advised to pay close attention to this example as it shows how a !(fin)

\begin{exmp}
  Let $X$ be the subset of $\mathbb{R}^2$ that is equal to
  \[
    \{ (x,y) \in \mathbb{R}^2 \mid -10 \leq x \leq 10 ; -1 < y < 1 \}.
  \]
  Now, bearing in mind our discussion from section
  \ref{sec:prelims:quotient}, we \emph{identify} the points $(-10,-y)$
  and $(10,y)$ for $y \in (-1,1)$. In other words we define the
  quotient space $X^*$ to consist of points $(x,y)$ in their own
  equivalence class for $-10 < x < 10$ and $-1 < y < 1$ while the
  points $(-10,-y)$ and $(10,y)$ are each in the same equivalence
  class. We call the latter class of points ``identified'' points.
  
  Now we will attempt to show that $X^*$ is indeed a surface. We will
  proceed by finding for each element $\mathbf{x}$ of $X^*$ an open
  neighbourhood characterised by a radius $r > 0$. Then we will show
  that our choice is homeomorphic to the unit ball on $\mathbb{R}^2$
  and finally we will show that given any two points the
  neighbourhoods can be chosen such that they don't intersect
  (Hausdorff condition).

  Assume that $\mathbf{x}$ isn't identified. This point is mapped by
  the projection map $\pi$ from a unique point in the interior of the
  rectangle $X \subseteq \mathbb{R}^2$. Working in this open rectangle
  as a metric space, we can find open balls given by radius $r >0$
  that is entirely contained in the interior of the rectangle. Now
  map this ball back to $X^*$ via $\pi$ and denote this set by
  $B_r(\mathbf{x}) \subseteq X^*$.
  
  Since the ball was contained in the interior of the rectangle, no
  point in them is identified, so the projection's restricton to the
  open ball in $X$ is invertible. It follows that $\pi^{-1}
  (B_r(\mathbf{x}))$ is exactly the ball that we started with and is
  open in $X$, so by definition of the quotient space
  $(B_r(\mathbf{x}))$ is an open neighbourhood of $\mathbf{x}$.

  In the case that $\mathbf{x}$ is an identified point, its pullback
  is the pair of points $\pm(10, y)$ on the border of $X$. Given $0 <
  r < \operatorname{min} \{ 1-y, -1-y \}$, let $U_r(\mathbf{x})$ be
  the intersection of the two open balls of radius $r$ about $\pm(10,
  y)$ with $X$. We define the image of $U_r(\mathbf{x})$ under the
  projection $\pi$ to be $B_r(\mathbf{x})$. Now it is easy to see that
  the preimage of $B_r(\mathbf{x})$ under $\pi$ is $U_r(\mathbf{x})$,
  which is open in the subspace $X$. It follows that $B_r(\mathbf{x})$
  suffices as a neighbourhood for $\mathbf{x}$.

  It is important to note however that we haven't defined quotient
  subspace $X^*$ and our notation $B_r(\mathbf{x})$ similar to that of
  open balls in metric spaces is only notional and will help with
  intuition later on.
\end{exmp}


\subsection{Stating the Classifiction Theorem}
\label{sec:surf:prototype}

Now we will show how some common surfaces can be written as quotient
spaces of various polygons by identifying edges. Then we will
introduce connected sums of surfaces and their respective quotient
space. Lastly we will prove an important lemma to simplify some
connected sums. Our goal in proving the classification theorem will
then be to show that an arbitrary surface is the quotient space of a
polygon as above. Finally with the ``prototype'' surfaces defined, we
will provide formal statement of the classification thoerem.

\subsubsection{Orientable Surfaces}

Here we describe the prototype surfaces that are orientable.





\subsection{Triangulation}
\label{sec:surf:triangulation}



\subsection{Adjoining a projective plane to a torus}
\label{sec:surf:adjoin}

\subsection{Forming a Polygon}
\label{sec:surf:polygon}

\subsection{Folding in pairs of the first kind}
\label{sec:surf:vertices}

\subsection{Identifying Vertices}
\label{sec:surf:folding}

\subsection{Rearranging pairs of the second kind}
\label{sec:surf:second}

\subsection{Final Arrangement}
\label{sec:surf:final}


%%% Local Variables:
%%% mode: latex
%%% TeX-master: "main"
%%% End:
