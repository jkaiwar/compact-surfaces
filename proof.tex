%% Tomorrow
%% surfaces example
%% extend to n manifold
%% orientability intuition
%% research definition of orientablity
%% prototype drawings and quotient space formations
%% connected sum definitions and n-torus quotient space formation
%% lemma 7.1


\section{Surfaces and their Classification}
\label{sec:surf}

We will finally be able to give a proper topological definition for
surfaces, and soon we will be able to state the theorem.

\subsection{Surfaces and Orientability}
\label{sec:surf:surfaces}

We begin by recalling the metric topology from example
\ref{exmp:metric}. With this topology in mind for the metric space
$\mathbb{R}^2$ under the euclidean metric, we have the following
definition

\begin{defn}
  We say that $S$ is a \emph{surface} if it is a Hausdorff space for
  which every point has a neighbourhood that is homeomorphic to the
  open unit ball $\{\mathbf{x} \in \mathbb{R}^2 \mid \lvert \mathbf{x}
  \rvert < 1 \}$.
\end{defn}

We often implicitly like to deal with surfaces that are
\emph{connected} and \emph{compact} as discussed in section
\ref{sec:prelims:compact}.

We present the following examples.

\begin{exmp}
  The plane $\mathbb{R}^2$ is trivially a surface; each point is
  contained in $\mathbb{R}^2$, an open set that is homeomorphic to the
  unit ball $B_1(0)$ via $f:B_1(0) \rightarrow \mathbb{R}^2$, given by
  \[
    x \mapsto \frac{x}{x^2 - 1}.
  \]

  It is easy to see as a simple extension that any open subset of
  $\mathbb{R}^2$ is a surface.
\end{exmp}


We will see many more examples of surfaces in section
\ref{sec:surf:prototype} but we will present one more example of a
surface to introduce the property of \emph{orientability} possessed by
connected surfaces. Here our example will be defined as a
\emph{quotient space}. The reader is advised to pay close attention to
this example as it shows how a !(fin).

\begin{exmp}
  \label{exmp:mobius}
  Let $X$ be the subset of $\mathbb{R}^2$ that is equal to
  \[
    \{ (x,y) \in \mathbb{R}^2 \mid -10 \leq x \leq 10 ; -1 < y < 1 \}.
  \]
  Now, bearing in mind our discussion from section
  \ref{sec:prelims:quotient}, we \emph{identify} the points $(-10,-y)$
  and $(10,y)$ for $y \in (-1,1)$. In other words we define the
  quotient space $X^*$ to consist of points $(x,y)$ in their own
  equivalence class for $-10 < x < 10$ and $-1 < y < 1$ while the
  points $(-10,-y)$ and $(10,y)$ are each in the same equivalence
  class. We call the latter class of points ``identified'' points.
  
  Now we will attempt to show that $X^*$ is indeed a surface. We will
  proceed by finding for each element $\mathbf{x}$ of $X^*$ an open
  neighbourhood characterised by a radius $r > 0$. Then we will show
  that our choice is homeomorphic to the unit ball on $\mathbb{R}^2$
  and finally we will show that given any two points the
  neighbourhoods can be chosen such that they don't intersect
  (Hausdorff condition).

  Assume that $\mathbf{x}$ isn't identified. This point is mapped by
  the projection map $\pi$ from a unique point in the interior of the
  rectangle $X \subseteq \mathbb{R}^2$. Working in this open rectangle
  as a metric space, we can find open balls given by radius $r >0$
  that is entirely contained in the interior of the rectangle. Now
  map this ball back to $X^*$ via $\pi$ and denote this set by
  $B_r(\mathbf{x}) \subseteq X^*$.
  
  Since the ball was contained in the interior of the rectangle, no
  point in them is identified, so the projection's restricton to the
  open ball in $X$ is invertible. It follows that $\pi^{-1}
  (B_r(\mathbf{x}))$ is exactly the ball that we started with and is
  open in $X$, so by definition of the quotient space
  $(B_r(\mathbf{x}))$ is an open neighbourhood of $\mathbf{x}$.

  In the case that $\mathbf{x}$ is an identified point, its pullback
  is the pair of points $\pm(10, y)$ on the border of $X$. Given $0 <
  r < \operatorname{min} \{ 1-y, -1-y \}$, let $U_r(\mathbf{x})$ be
  the intersection of the two open balls of radius $r$ about $\pm(10,
  y)$ with $X$. We define the image of $U_r(\mathbf{x})$ under the
  projection $\pi$ to be $B_r(\mathbf{x})$. Now it is easy to see that
  the preimage of $B_r(\mathbf{x})$ under $\pi$ is $U_r(\mathbf{x})$,
  which is open in the subspace $X$. It follows that $B_r(\mathbf{x})$
  suffices as a neighbourhood for $\mathbf{x}$.

  It is important to note however that we haven't defined quotient
  subspace $X^*$ and our notation $B_r(\mathbf{x})$ similar to that of
  open balls in metric spaces is only notional and will help with
  intuition later on.
\end{exmp}

!(use cut and paste language)

\subsection{Triangulation}
\label{sec:surf:triangulation}

We define what it means to triangulate a surface.

\begin{defn}
  Given a surface $S$, we say that the collection of closed sets $\{
  T_1,  T_2, \dots T_n \}$ \emph{triangulates} $S$ if it covers $S$
  and for each $i = 1,2, \dots, n$ there exists homoeomorphisms $\phi
  : T_i' \rightarrow T_i$ where each $T_i'$ is a triangle in
  $\mathbb{R}^2$ under the metric topology (described in example
  \ref{exmp:metric}). We say that the \emph{edges} of $T_i$ are images
  of the edges of $T_i'$ for each $i$; similarly the \emph{vertices}
  of $T_i$ are the images of the vertices of $T_i'$.
\end{defn}

We take it as a fact that any compact surface can be triangulated. We
understand that this is the case from !(complete).

Given that an arbitrary surface $S$ can be triangulated, we make
``cuts'' along the edges of the triangles and notice that each
triangle can be ``pasted'' onto the plane $\mathbb{R}^2$. Since
translations are homeomorphisms as well, we can translate the
triangles appropriately and have all triangles from the triangulation
``pasted'' on the plane in a manner by which they do not intersect
each other.

More formally, we begin by inductively relabelling our
triangulation. Suppose we have $n$ ``triangles'' that cover $S$; then
we proceed to label them inductively starting by labelling any
triangle $T_1$. Then, for each $1 \leq i \leq n-1$, choose a $T_{i+1}$
that has an edge $e_i$ in common with one of the preceeding
triangles $T_1, T_2, \dots, T_i$. Such a triangle must always exist by
our implicit assumtion of the connectedness of $S$. Numbering  our
triangle in this manner, we have defined the edges $e_i$ for $2 \leq i
\leq n$. 

For each triangle $T_i$ we must by definition have a homeomporphism
$\psi_i: T_i' \rightarrow T_i$, where $T_i'$ is a triangle in
$\mathbb{R}^2$. Translating the triangles $T_i'$ to $T_i''$ such that
they are pairwise disjoint, it is clear that there exists
homeomorphisms $\phi_i: T_i'' \rightarrow T_i$. Finally, given $T'' =
\cup_{i=1}^n T_i''$ and $\phi : T'' \rightarrow S$ defined via $\phi
\restriction T_i = \phi_i$ it is easy to see that $\phi : T''
\rightarrow S$ is a homeomorphism. 

For now, we assume that our surface $S$ doesn't have a
boundary and as a result each triangle edge has a corresponding edge %% what is a boundary?
on some other triangle that it is identified with.

Here we have explained how an arbitrary compact surface is
homeomorphic to finitely many triangles pasted on the plane
\emph{under the quotient topology} identifying pairs of edges. It is
important to note that we must include the added information of which
edges are identified with each other when we represent this quotient
space, and also include in what direction they are to be glued on. In
particular we will represent this quotient space as the triangle that
forms it with each edge labeled and an it's identification
``direction''. Figure \ref{fig:represent} shows us how we may imagine
this space.

\begin{figure}[h]
  \centering
  \includegraphics[width=3cm]{pepe.jpg}
  \caption{Representing our quotient topology and ``gluing''
    corresponding edges.}
  \label{fig:represent}
\end{figure}

We will show that the triangles of this quotient space can be glued
onto one another so that we end up with a polygon.

%% work in 2.4 (a)
\subsection{Stating the Classifiction Theorem}
\label{sec:surf:prototype}

Now we will show how some common surfaces can be written as quotient
spaces of various polygons by identifying edges. Then we will
introduce connected sums of surfaces and their respective quotient
space. Lastly we will prove an important lemma to simplify some
connected sums. Our goal in proving the classification theorem will
then be to show that an arbitrary surface is the quotient space of a
polygon as above. Finally with the ``prototype'' surfaces defined, we
will provide formal statement of the classification thoerem.

\begin{thm}
  Any compact surface is either homeomorphic to a sphere, or to a
  connected sum of tori, or to a connected sum of projective planes.
\end{thm}

\subsubsection{Orientable Surfaces}

Here we describe the prototype surfaces that are orientable, namely
the sphere and n-tori.


\subsection{Adjoining a projective plane to a torus}
\label{sec:surf:adjoin}

\subsection{Forming a Polygon}
\label{sec:surf:polygon}

\subsection{Folding in pairs of the first kind}
\label{sec:surf:vertices}

\subsection{Identifying Vertices}
\label{sec:surf:folding}

\subsection{Rearranging pairs of the second kind}
\label{sec:surf:second}

\subsection{Final Arrangement}
\label{sec:surf:final}


%%% Local Variables:
%%% mode: latex
%%% TeX-master: "main"
%%% End:
