\section{Topological Preliminaries}
\label{sec:prelims}

Recall, that for a metric space $(X,d)$ we have that a subset $A
\subseteq X$ is open if for every $x_0 \in A$, there exists $\delta >
0$ such that $\{x \in X | d(x,x_0) < \delta\} \subseteq A$.

Many properties of metric spaces can be defined even without
describing an explicit metric and we can talk about limits, continuity
and compactness simply by considering ``open'' sets. It seems then
that we can generalise many results from metric spaces in a space
where we start by defining what sets are open. That is exactly what we
do when we define topological spaces.

% Comment on not a full discussion
\subsection{Topological Spaces}
\label{sec:prelims:topospace}

\begin{defn}
  Let $X$ be a set and suppose $\mathscr{T}$ is a collection of
  subsets of $X$ that satisfies
  \begin{enumerate}
  \item $X$ and $\emptyset$ belong to $\mathscr{T}$;
  \item Given a subcollection $\{A_\alpha \in \mathscr{T} | \alpha \in
    I$, where $I$ is some indexing set, we have that the union
    $\cup_{\alpha \in I}A_\alpha$ belongs to $\mathscr{T}$ as well;
  \item Given a finite subcollection $\{B_i \in \mathscr{T} | i =
    1,2,3,\dots, n \}$ for some $n \in \mathbb{N}$ we have that the
    intersection $\cap_{i = 1}^{n}$ belongs to $\mathscr{T}$ as well.
  \end{enumerate}
  Then we say that $\mathscr{T}$ is a topology on $X$ and that every
  element of $\mathscr{T}$ is an \emph{open} set. We say that
  $(X,\mathscr{T})$ is a topological space.
\end{defn}

Immediately we have a couple of trivial examples of topologies given
any set $X$.

\begin{exmp}
  \label{exmp:trivial}
  The smallest collection of subsets satisfying the above axioms is
  clearly $\{ X, \emptyset \}$. We call this the \emph{trivial
    topology} on $X$. It is the \emph{coarsest} topology on $X$, i.e.\
  it is containted in every topology on $X$, by the first axiom.
\end{exmp}

\begin{exmp}
  \label{exmp:discrete}
  The power set on $X$, i.e.\ the collection of all subsets of $X$,
  including $X$ trivially satisfies all the axioms. We call it the
  \emph{discrete topology} on $X$ and notice that it is the
  \emph{finest} topology on $X$, that is, it contains every other
  topology on $X$.
\end{exmp}

Since we have referred to notions of \emph{fineness} or
\emph{coarseness} with respect to topology without a proper
exposition, it is worth noting that not all topologies can be
compared; in other words, we might come across two topologies on a set
$X$ where one isn't contained in the other. !(candidate for footnote)


Consider the following more involved example.

\begin{exmp}
  \label{exmp:countable}
  We have that $\mathscr{T}_C := \{U \subset X \mid X \setminus U
  \text{ countable or all of } X \}$ is a topology on $X$.
\end{exmp}

\begin{proof} %%needs correction?
  We show each axiom of topological spaces for $\mathscr{T}_C$.
  \begin{enumerate}
  \item Clearly $\emptyset \in \mathscr{T}_C$ since $X \setminus
    \emptyset = X$, that is, all of $X$.
  \item Let $S$ be a subcollection of $\mathscr{T}_C$ indexed by some
    set $I$; $S := \{U_i \in \mathscr{T}_C \mid i \in I \}$. Now, we
    want to show that $\cup_{i \in I} U_i \in \mathscr{T}_C$. We note
    that $X \setminus \cup_{i \in I} U_i = \cap_{i \in I}(X \setminus
    U_i) \subset (X \setminus U_j)$, for some arbitrary $j \in
    I$. However $(X \setminus U_j)$ is countable by hypothesis and so
    must be $X \setminus \cup_{i \in I} U_i$, its subset, and the
    result follows.
  \item Let $n \in \mathbb{N}$, a finite positive integer and suppose
    that $V_1, \dots, V_n$ is a finite subcollection of open sets. We
    want to show that their intersection is open. We have that
    \[
      X \setminus (\cap_{k=1}^n V_k) = \cup_{k=1}^n (X \setminus V_k).
    \]
    Since each $V_k$ is open, it's complement is countable. We know
    that finite unions of countable sets are countable (from
    introduction to proofs) so $X \setminus \cap_{k=1}^n V_k$ is   %% refering this
    countable and the finite union is open by definition.
  \end{enumerate}
\end{proof}

On the other hand this leads to an instructive counterexample.

\begin{exmp}
  The collection
  \[
    \mathscr{T}_\infty = \{U \mid X \setminus U \text{ is infinite or
      empty or all of } X \}
  \]
  is \emph{not} a topology on $X$
\end{exmp}

\begin{proof}
  $\mathscr{T}_\infty$ is not necessarily a topology because in second
axiom, with the indexing as in example \ref{exmp:countable}, $X
\setminus \cup_{i \in I} U_i = \cap_{i \in I} (X \setminus U_i)$ may
be finite.

  For example if $X = \mathbb{Z}$, the union of the open sets $\{ \{1,
2, 3, \dots \}$ and $\{-1, -2, -3, \dots \} \}$, we have that the
union of these sets has complement $(X \setminus \{1, 2, 3, \dots \})
\cap (\setminus \{-1, -2, -3, \dots \}) = \{0, -1, -2, \dots \} \cap
\{0, 1, 2, \dots \} = \{ 0 \}$ which is finite and nonempty and so the
union is not open in $\mathscr{T}_\infty$.
\end{proof}


\subsubsection{Basis for a Topology}
\label{sec:prelims:topospace:basis}

To look at some more sophisticated examples of topologies !(reeval), we will
first introduce the notion of \emph{basis} in topological spaces. Like
in Linear Algebra, we find that the \emph{basis}  generates our
topological space. 

\begin{defn}
  Given a set $X$, we say that a collection of subsets $\mathscr{B}$
  is a \emph{basis} if it satisfies
  \begin{enumerate}
  \item For each $x \in X$ there exists a \emph{basis element} $B \in
    \mathscr{B}$ with $x \in B$.
  \item If $x \in B_1 \cap B_2$, where $B_1$ and $B_2$ are basis
    elements, then we have that there exists a basis element $B$ such
    that $B$ both contains $x$ and is contained in the intersection
    $B_1 \cap B_2$.
  \end{enumerate}
  Given the basis $\mathscr{B}$ as above we say that $U$ is open in
  the \emph{topology generated by} $\mathscr{B}$ if for each $x \in U$
  there exists some basis element $B$ with $x \in B \subseteq U$.
\end{defn}

To see that such a collection generated as such is indeed a topology
is trivial and ommitted here !(ref).

We have the following Lemmas to better see how bases generate their
respective topologies and how bases can be identified.

\begin{lem}
  \label{lem:union}
  Let $\mathscr{B}$ be a basis on a set $X$. Then the topology
  $\mathscr{T}$ generated by $\mathscr{B}$ is equal to the collection
  of all unions on the elements of $\mathscr{B}$.
\end{lem}

\begin{proof}
  Any collection of elements in $\mathscr{B} \subseteq \mathscr{T}$
  must have their union open by the second axiom of topological
  spaces. Given an open set $S$ and $x \in S$ there exists a basis
  element $B_x$ with $x \in B_x \subseteq S$. Therefore $S$ must
  contain the union $\cup_{x \in S} B_x$. Conversely we have that each
  element $x$ of $S$ is contained in some $B_x$, thus $S$ is contained
  in the union and we have that $S = \cup_{x \in S} B_x$ as required
\end{proof}

From this proof it becomes clear that unlike linear algebra, we may
have our open set generated by basis elements in multiple
ways. !(example).

\begin{lem}
  \label{lem:collection}
  Let $(X,\mathscr{T})$ be a topological space. If $\mathscr{C}$ is a
  collection of open sets such that for each open set $U$ and $x \in
  U$, we have that there exists $C \in \mathscr{C}$ with 
  \[
    x \in C \subseteq U.
  \]
  Then $\mathscr{C}$ is a basis for $\mathscr{T}$.
\end{lem}

\begin{proof}
  We will show that $\mathscr{C}$ is a basis when the above condition
  is met. Since $X$ is open, we can apply the condition $x \in C
  \subseteq X$ for each $x \in X$ to give us (1) of the definition for
  the Basis.

  For the second condition suppose $x \in C_1 \cap C_2$ where $C_1$
  and $C_2$ belong to the open collection $\mathscr{C}$. $x \in C_1
  \cap C_2$ must be open as well by the third axiom so applying our
  condition to the intersection we have that there exists $C \in
  \mathscr{C}$ with
  \[
    x \in C \subseteq x \in C_1 \cap C_2
  \]
  as required. Thus $\mathscr{C}$ is a basis, but we need to show that
  it generates $\mathscr{T}$.

  Suppose $\mathscr{T}'$ is generated by $\mathscr{C}$. We have by
  assumption that any set $U \in \mathscr{T}$ has that for all $x \in
  U$ there exists $C\ in \mathscr{C}$ such that $x \in C \subseteq U$
  and thus $U \in \mathscr{T}'$ by definition.

  Conversely any element $U'$ of $\mathscr{T}'$ can be written as a
  union of elements of $\mathscr{C}$ by Lemma \ref{lem:union}. But
  $\mathscr{C}$ consists of sets in $\mathscr{T}$ and thus the union
  $U'$ is in $\mathscr{T}$ by the second axiom. Therefore we have
  $\mathscr{T}' = \mathscr{T}$.
\end{proof}

\begin{exmp}
  \label{exmp:standard}
  If we say that for all $a<b$, $(a,b) = \{x | a < x <b \}$ is an open
  interval, the \emph{standard topology} on $\mathbb{R}$ is defined to be
  generated by the basis $\mathscr{B}$ consisting of all open
  intervals:
  \[
    \mathscr{B} = \{(a,b)| -\infty \leq a < b \leq \infty\}.
  \]
\end{exmp}



\subsection{The Subspace Topology}
\label{sec:prelims:subspace}

\subsection{Closed Sets}
\label{sec:prelims:closed}

\subsection{Continuity}
\label{sec:prelims:continuity}

\subsection{The Quotient Topology}
\label{sec:prelims:quotient}


%%% Local Variables:
%%% mode: latex
%%% TeX-master: "main"
%%% End:
