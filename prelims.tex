\section{Topological Preliminaries}
\label{sec:prelims}

Recall, that for a metric space $(X,d)$ we have that a subset $A
\subseteq X$ is open if for every $x_0 \in A$, there exists $\delta >
0$ such that $\{x \in X | d(x,x_0) < \delta\} \subseteq A$.

Many properties of metric spaces can be defined even without
describing an explicit metric and we can talk about limits, continuity
and compactness simply by considering ``open'' sets. It seems then
that we can generalise many results from metric spaces in a space
where we start by defining what sets are open. That is exactly what we
do when we define topological spaces.

% Comment on not a full discussion
\subsection{Topological Spaces}
\label{sec:prelims:topospace}

\begin{defn}
  Let $X$ be a set and suppose $\mathscr{T}$ is a collection of
  subsets of $X$ that satisfies
  \begin{enumerate}
  \item $X$ and $\emptyset$ belong to $\mathscr{T}$;
  \item Given a subcollection $\{A_\alpha \in \mathscr{T} | \alpha \in
    I$, where $I$ is some indexing set, we have that the union
    $\cup_{\alpha \in I}A_\alpha$ belongs to $\mathscr{T}$ as well;
  \item Given a finite subcollection $\{B_i \in \mathscr{T} | i =
    1,2,3,\dots, n \}$ for some $n \in \mathbb{N}$ we have that the
    intersection $\cap_{i = 1}^{n} B_i$ belongs to $\mathscr{T}$ as well.
  \end{enumerate}
  Then we say that $\mathscr{T}$ is a topology on $X$ and that every
  element of $\mathscr{T}$ is an \emph{open} set. We say that
  $(X,\mathscr{T})$ is a topological space.
\end{defn}

Immediately we have a couple of trivial examples of topologies given
any set $X$.

\begin{exmp}
  \label{exmp:trivial}
  The smallest collection of subsets satisfying the above axioms is
  clearly $\{ X, \emptyset \}$. We call this the \emph{trivial
    topology} on $X$. It is the \emph{coarsest} topology on $X$, i.e.\
  it is containted in every topology on $X$, by the first axiom.
\end{exmp}

\begin{exmp}
  \label{exmp:discrete}
  The power set on $X$, i.e.\ the collection of all subsets of $X$,
  including $X$ trivially satisfies all the axioms. We call it the
  \emph{discrete topology} on $X$ and notice that it is the
  \emph{finest} topology on $X$, that is, it contains every other
  topology on $X$.
\end{exmp}

Since we have referred to notions of \emph{fineness} or
\emph{coarseness} with respect to topology without a proper
exposition, it is worth noting that not all topologies can be
compared; in other words, we might come across two topologies on a set
$X$ where one isn't contained in the other. !(candidate for footnote)


Consider the following more involved example.

\begin{exmp}
  \label{exmp:countable}
  We have that $\mathscr{T}_C := \{U \subset X \mid X \setminus U
  \text{ countable or all of } X \}$ is a topology on $X$.
\end{exmp}

\begin{proof} %%needs correction?
  We show each axiom of topological spaces for $\mathscr{T}_C$.
  \begin{enumerate}
  \item Clearly $\emptyset \in \mathscr{T}_C$ since $X \setminus
    \emptyset = X$, that is, all of $X$.
  \item Let $S$ be a subcollection of $\mathscr{T}_C$ indexed by some
    set $I$; $S := \{U_i \in \mathscr{T}_C \mid i \in I \}$. Now, we
    want to show that $\cup_{i \in I} U_i \in \mathscr{T}_C$. We note
    that $X \setminus \cup_{i \in I} U_i = \cap_{i \in I}(X \setminus
    U_i) \subset (X \setminus U_j)$, for some arbitrary $j \in
    I$. However $(X \setminus U_j)$ is countable by hypothesis and so
    must be $X \setminus \cup_{i \in I} U_i$, its subset, and the
    result follows.
  \item Let $n \in \mathbb{N}$, a finite positive integer and suppose
    that $V_1, \dots, V_n$ is a finite subcollection of open sets. We
    want to show that their intersection is open. We have that
    \[
      X \setminus (\cap_{k=1}^n V_k) = \cup_{k=1}^n (X \setminus V_k).
    \]
    Since each $V_k$ is open, it's complement is countable. We know
    that finite unions of countable sets are countable (from
    introduction to proofs) so $X \setminus \cap_{k=1}^n V_k$ is   %% refering this
    countable and the finite union is open by definition.
  \end{enumerate}
\end{proof}

On the other hand this leads to an instructive counterexample.

\begin{exmp}
  The collection
  \[
    \mathscr{T}_\infty = \{U \mid X \setminus U \text{ is infinite or
      empty or all of } X \}
  \]
  is \emph{not} a topology on $X$
\end{exmp}

\begin{proof}
  $\mathscr{T}_\infty$ is not necessarily a topology because in second
axiom, with the indexing as in example \ref{exmp:countable}, $X
\setminus \cup_{i \in I} U_i = \cap_{i \in I} (X \setminus U_i)$ may
be finite.

  For example if $X = \mathbb{Z}$, the union of the open sets $\{ \{1,
2, 3, \dots \}$ and $\{-1, -2, -3, \dots \} \}$, we have that the
union of these sets has complement $(X \setminus \{1, 2, 3, \dots \})
\cap (\setminus \{-1, -2, -3, \dots \}) = \{0, -1, -2, \dots \} \cap
\{0, 1, 2, \dots \} = \{ 0 \}$ which is finite and nonempty and so the
union is not open in $\mathscr{T}_\infty$.
\end{proof}


\subsubsection{Basis for a Topology}
\label{sec:prelims:topospace:basis}

To look at some more sophisticated examples of topologies !(reeval), we will
first introduce the notion of \emph{basis} in topological spaces. Like
in Linear Algebra, we find that the \emph{basis}  generates our
topological space. 

\begin{defn}
  Given a set $X$, we say that a collection of subsets $\mathscr{B}$
  is a \emph{basis} if it satisfies
  \begin{enumerate}
  \item For each $x \in X$ there exists a \emph{basis element} $B \in
    \mathscr{B}$ with $x \in B$.
  \item If $x \in B_1 \cap B_2$, where $B_1$ and $B_2$ are basis
    elements, then we have that there exists a basis element $B$ such
    that $B$ both contains $x$ and is contained in the intersection
    $B_1 \cap B_2$.
  \end{enumerate}
  Given the basis $\mathscr{B}$ as above we say that $U$ is open in
  the \emph{topology generated by} $\mathscr{B}$ if for each $x \in U$
  there exists some basis element $B$ with $x \in B \subseteq U$.
\end{defn}

To see that such a collection generated as such is indeed a topology
is trivial and ommitted here !(ref).

We have the following Lemmas to better see how bases generate their
respective topologies and how bases can be identified.

\begin{lem}
  \label{lem:union}
  Let $\mathscr{B}$ be a basis on a set $X$. Then the topology
  $\mathscr{T}$ generated by $\mathscr{B}$ is equal to the collection
  of all unions on the elements of $\mathscr{B}$.
\end{lem}

\begin{proof}
  Any collection of elements in $\mathscr{B} \subseteq \mathscr{T}$
  must have their union open by the second axiom of topological
  spaces. Given an open set $S$ and $x \in S$ there exists a basis
  element $B_x$ with $x \in B_x \subseteq S$. Therefore $S$ must
  contain the union $\cup_{x \in S} B_x$. Conversely we have that each
  element $x$ of $S$ is contained in some $B_x$, thus $S$ is contained
  in the union and we have that $S = \cup_{x \in S} B_x$ as required
\end{proof}

From this proof it becomes clear that unlike linear algebra, we may
have our open set generated by basis elements in multiple
ways. !(example).

\begin{lem}
  \label{lem:collection}
  Let $(X,\mathscr{T})$ be a topological space. If $\mathscr{C}$ is a
  collection of open sets such that for each open set $U$ and $x \in
  U$, we have that there exists $C \in \mathscr{C}$ with 
  \[
    x \in C \subseteq U.
  \]
  Then $\mathscr{C}$ is a basis for $\mathscr{T}$.
\end{lem}

\begin{proof}
  We will show that $\mathscr{C}$ is a basis when the above condition
  is met. Since $X$ is open, we can apply the condition $x \in C
  \subseteq X$ for each $x \in X$ to give us (1) of the definition for
  the Basis.

  For the second condition suppose $x \in C_1 \cap C_2$ where $C_1$
  and $C_2$ belong to the open collection $\mathscr{C}$. $x \in C_1
  \cap C_2$ must be open as well by the third axiom so applying our
  condition to the intersection we have that there exists $C \in
  \mathscr{C}$ with
  \[
    x \in C \subseteq x \in C_1 \cap C_2
  \]
  as required. Thus $\mathscr{C}$ is a basis, but we need to show that
  it generates $\mathscr{T}$.

  Suppose $\mathscr{T}'$ is generated by $\mathscr{C}$. We have by
  assumption that any set $U \in \mathscr{T}$ has that for all $x \in
  U$ there exists $C\ in \mathscr{C}$ such that $x \in C \subseteq U$
  and thus $U \in \mathscr{T}'$ by definition.

  Conversely any element $U'$ of $\mathscr{T}'$ can be written as a
  union of elements of $\mathscr{C}$ by Lemma \ref{lem:union}. But
  $\mathscr{C}$ consists of sets in $\mathscr{T}$ and thus the union
  $U'$ is in $\mathscr{T}$ by the second axiom. Therefore we have
  $\mathscr{T}' = \mathscr{T}$.
\end{proof}

\begin{exmp}
  \label{exmp:standard}
  If we say that for all $a<b$, $(a,b) = \{x | a < x <b \}$ is an open
  interval, the \emph{standard topology} on $\mathbb{R}$ is defined to be
  generated by the basis $\mathscr{B}$ consisting of all open
  intervals:
  \[
    \mathscr{B} = \{(a,b)| -\infty \leq a < b \leq \infty\}.
  \]
\end{exmp}

\begin{exmp}
  \label{exmp:metric}
  Example \ref{exmp:standard} hints at a more general example. We
  introduce the \emph{metric topolgy} as the topology generated by all
  open balls \footnote{Recall that an open ball about a point $x_0$
    and with diameter $\delta$ in a metric space is the set of all
    points with distance less than $\delta$ from $x_0$} in a metric
  space. This topological space is indeed consistent with the
  definition of open sets (of a metric space) that we recalled at the
  beginning of this section.
\end{exmp}

\subsection{The Subspace Topology}
\label{sec:prelims:subspace}

We now turn our attention to the subspace topology which will be
important in our discussion of the quotient topology.

\begin{defn}
  Let $(X, \mathscr{T})$ be a topological space. For $Y \subseteq X$,
  define
  \[
    \mathscr{T}_Y = \{ Y \cap U | U \in \mathscr{T} \}
  \]
  to be the \emph{subspace topology} on $Y$.
\end{defn}

We need to show that this is indeed a topology. We will use the terms
\emph{open in $X$} for $\mathscr{T}$ and \emph{open in $Y$} for the
subspace topology.

\begin{proof}
  We verify the axioms as usual.
  \begin{enumerate}
  \item $Y \cap \emptyset = \emptyset$ and $Y \cap X = Y$.
  \item For a collection $\{U_i\}_{i \in I}$, we have that
    \[
      \cup_{i \in I}(Y \setminus U_i) = Y \setminus (\cap_{i \in I} U_i)
    \]
    Since $\cap_{i \in I} U_i$ is open by the second axiom in $X$, $Y
    \setminus \cap_{i \in I} U_i$ is open in $Y$ by definition.
  \item For a finite collection $\{V_i\}_{i=1}^n$, $n \in \mathbb{N}$,
    we have
    \[
      \cap_{i=1}^n (Y \cap V_i) = Y \cap (\cap_{i=1}^n V_i).
    \]
    But $\cap_{i=1}^n V_i$ is open in $X$ by the third axiom and we
    are done.
  \end{enumerate}
\end{proof}

The basis for a the subspace topology is found to be analogous.

\begin{lem}
  \label{lem:subspace}
  If $\mathscr{B}$ is a basis for the topology on $X$, then
  \[
    \mathscr{B}_Y = \{B \cap Y | B \in \mathscr{B}\}
  \]
  is a basis for $Y$ as a subspace of $X$.
\end{lem}

\begin{proof}
  Let $U \subset X$ be open and $y \in U \cap Y$. We know that there
  exists $B \in \mathscr{B}$ such that $y \in (B \cap Y) \subset (U
  \cap Y)$. By Lemma \ref{lem:collection} we are done.
\end{proof}

A helpful lemma. %% ask... style

\begin{lem}
  Let $Y$ be a subspace of $X$. If $U$ is an open set in $Y$ and $Y$
  itself is open in $X$, then $U$ is open in $X$.
\end{lem}

\begin{proof}
  Since $U \subseteq Y$ is open, there exists some $V$ that is open in
  $X$ such that $U = V \cap Y$. The third axiom affirms that this
  finite intersection is open in $X$. 
\end{proof}

%% We will now look at some examples.

\subsection{Closed Sets}
\label{sec:prelims:closed}

Now we move on to a discussion about closed sets and limit points. The
definitions we make here will be crucial to our definition of surfaces
and will be used in developing our theory on continuous functions in
section \ref{sec:prelims:continuity}.

\begin{defn}
  Suppose $(X,\mathscr{T}$ is a topological space. We say that a
  subset $A \subseteq X$ is closed if $X \setminus A$ is open.
\end{defn}

We present an immediate theorem that illustrates analogs with the
properties of \emph{open} sets.

\begin{thm}
  Let $(X,\mathscr{T})$ be a topological space. Then we have the
  following. 
  \begin{enumerate}
  \item $\emptyset$ and $X$ are closed.
  \item Arbitrary intersections of closed sets are closed.
  \item Finite unions of closed sets are closed.
  \end{enumerate}
\end{thm}

\begin{proof}
  \begin{enumerate}
  \item $\emptyset = X \setminus X$ and $X = X \setminus \emptyset$.
  \item If we have a collection of closed sets $\{ U_i \mod i \in I
    \}$ where $I$ is some indexing set and we have, by De Morgan's
    laws,
    \[
      X \setminus \left( \bigcap_{i \in I}U_i \right) = \bigcup_{i \in I} (X
      \setminus U_i).
    \]
    But each $X \setminus U_i$ is open by definition and so is their
    union by the second axiom of topological spaces. The result
    follows by taking the complement of the left hand side of the
    above equation, which must be closed.
  \item If $V_1, \dots, V_n$ is a finite collection of closed subsets
    of $X$ ($n \in \mathbb{N}$), we have that
    \[
      X \setminus \left( \bigcup_{i=1}^n V_i \right) = \bigcap_{i =
        1}^n (X \setminus V_i)
    \]
    is open by the third axiom. Taking the complement of the left hand
    side gives us that the finite union is closed.
  \end{enumerate}
\end{proof}

Now we have a theorem to describe closed sets in the subspace
topology.

\begin{thm}
  \label{thm:closed-in-subspace}
  Let $Y$ be a subspace of the topological space $X$. Then $A
  \subseteq Y$ is closed in $Y$ \emph{if and only if} it is the
  intersection of a set closed in $X$ with $Y$
\end{thm}

\begin{proof}
  For the sufficient condition, let $A = C \cap Y$, where $C$ is
  closed in $X$. By definition, $X \setminus C$ is open and so $(X
  \setminus C) \cap Y$ is open in the subspace $Y$. Now
  \[
    (X \setminus C) \cap Y = (X \cap Y) \setminus (C \cap Y) = Y
    \setminus A,
  \]
  is open and therefore $A$ is closed in the subspace $Y$.

  For the neccessary condition, Let $A \subseteq Y$ be closed. Then $Y
  \setminus A$ is open in $Y$. By definition of the subspace topology,
  there exists a $V$ that is open in $X$ such that $V \cap Y = Y
  \setminus A$. We will show that $A$ is the intersection of the
  closed $X \setminus V$ with $Y$.
  \[
    (X \setminus V) \cap Y = (X \cap Y) \setminus V = Y \setminus V = A
  \]
\end{proof}

It will be of interest to us for each set to study the \emph{smallest}
closed set containing it and the \emph{largest} open set it contains;
we make the following definitions.

\begin{defn}
  Let $(X, \mathscr{T})$ is a topological space and $A \subseteq
  X$. We define
  \begin{itemize}
  \item The \emph{interior} of $A$ to be
    \[
      \operatorname{int}(A) = \bigcup_{V \subseteq A} V,
    \]
    where every $V$ is open.
  \item The \emph{closure} of $A$,
    \[
      \overline{A} = \bigcap_{W \supseteq A} W,
    \]
    where $W$ is closed in $X$.
  \end{itemize}
\end{defn}

It is easy to deduce that $\operatorname{int}(A)$ is open and that
$\overline{A}$ is closed for any set $A$. Furthermore we have that
\[
  \operatorname{int}(A) \subseteq A \subseteq \overline{A}.
\]

%% comment

\begin{thm}
  Let $Y$ be a subspace of a topological space $X$, and let $A
  \subseteq Y$. If $\overline{A}$ is the closure of $A$ in $X$,
  $\overline{A} \cap Y$ is the closure of $A$ in $Y$.
\end{thm}

\begin{proof}
  Let $B$ denote the closure of $A$ in $Y$. If the set $\overline{A}$
  is closed in $X$, then $\overline{A} \cap Y$ is closed in $Y$ by
  Theorem \ref{thm:closed-in-subspace}. We also have $A \subseteq
  (\overline{A} \cap Y)$ and $(\overline{A} \cap Y) \subseteq
  \overline{A}$. By the definition of $B$, we have $B \subseteq
  (\overline{A} \cap Y)$. Now since $B$ is closed in $Y$ by Theorem
  \ref{thm:closed-in-subspace}, there exists a set $C$ that is closed
  in $X$ such that $B = C \cap Y$. But $A \subseteq B$ and so $A
  \subseteq C$. Since $\overline{A}$ is the smallest closed subset
  containing $A$, we must have $\overline{A} \subseteq C$. Therefore
  \[
    (\overline{A} \cap Y) \subseteq (C \cap Y) = B
  \]
  and we have equality as required.
\end{proof}

Now we introduce a pair of useful equivalant conditions for a set to
be a closure.

\begin{thm}
  \label{thm:closure-equivalent}
  Let $(X, \mathscr{T})$ be a topological space and $A \subseteq
  X$. Let $\mathscr{B}$ be a basis for $\mathscr{T}$. We have the
  following.
  \begin{enumerate}
  \item $x \in \overline{A}$ \emph{if and only if} every
    neighbourhood \footnote{we say that a set $U$ is a
      \emph{neighbourhood} of a point $x$ if $U$ is open and contains
      $x$.} of $x$ intersects $A$.
  \item $x \in \overline{A}$ \emph{if and only if} every
    basis element containing $x$ intersects $A$.
  \end{enumerate}
\end{thm}

\begin{proof}
  \begin{enumerate}
  \item We will show that $x \not\in \overline{A} \iff$ there exists a
    neighbourhood of $x$ that doesn't intersect $A$.

    For the neccessary condition suppose $x \not\in
    \overline{A}$. Naturally $x$ belongs to the set complement of
    $\overline{A}$ which is a open and doesn't intersect $A$, as
    required.
    
    For the sufficient condition, suppose $U$ is a neighbourhood
    of $x$ that doesn't intersect $A$. Now $X \setminus U$ is closed
    and contains $A$ so it therefore contains $\overline{A}$. But $U$
    contains $x$ so $x \not\in (X \setminus U) \subseteq
    \overline{A}$.
  \item The neccessary condition is just an application of (1), since
    basis elements are always open

    For the sufficient condition suppose every basis element
    containing $x$ intersects $A$. If $U$ is a neighbourhood of $x$,
    let $B \in \mathscr{B}$ such that $x \in B \subseteq U$ by the
    definition of the basis. Since $B$ intersects $A$, $U$ must as
    well. Now the sufficient condition of (1) gives us that $x \in
    \overline{A}$ 
  \end{enumerate}
\end{proof}

\begin{defn}
  If $(X, \mathscr{T})$ is a topological space and $A \subseteq X$, we
  say that $x \in X$ is a \emph{limit point} of $A$ if every
  neighbourhood of $x$ intersects $A \setminus \{ x \}$. 
\end{defn}

%% comment

\begin{thm}
  \label{thm:limit-closure}
  Suppose $(X,\mathscr{T})$ is a topological space and $A \subseteq
  X$. Let $A'$ be the set of all limit points of $A$. Then
  \[
    \overline{A} = A \cup A'.
  \]
\end{thm}

\begin{proof}
  Let $x \in \overline{A}$, then any neighbourhood of $x$ intersects
  $A$ by Theorem \ref{thm:closure-equivalent}. If $x \in A$ we have $x
  \in (A \cup A')$. If $x \not\in A$, the intersection of each of its
  neighbourhoods with $A \setminus \{ x \} = A$ is still nonempty,
  thus $x \in A' \subseteq A \cup A'$. 

  Conversely let $x \in A \cup A'$, If $x \in A$ we are trivially done
  since $A \subseteq \overline{A}$, so assume $x \in A' \setminus
  A$. Thus $x$ is a limit point of $A$ and each of its neighbourhoods
  intersect $A$. By Theorem \ref{thm:closure-equivalent}, $x \in
  \overline{A}$.
\end{proof}

\begin{defn}
  We say that a topological space $(X, \mathscr{T})$ is
  \emph{Hausdorff} if for each pair $x_1$, $x_2$ there exist
  neighbourhoods of $x_1$ and $x_2$ that are disjoint.
\end{defn}

\subsection{Continuity}
\label{sec:prelims:continuity}

We are already familiar with $\epsilon-\delta$ continuity as
encountered in real analysis and metric spaces.

\begin{defn}
  Let $(X,d)$ and $(Y,e)$ be metric spaces. We say that $f: X
  \rightarrow Y$ is continuous if at every $x_0 \in X$, for all
  $\epsilon > 0$, there exists $\delta > 0$ such that for each $x \in
  X$
  \[
    d(x_0,x) < \delta \implies e(f(x_0),f(x)) < \epsilon.
  \]
\end{defn}

This leads us to a the open set condition for continuity which allows
us to define continuity purely in terms of open sets, discarding the
metric. It is stated below without proof. To see a proof for this
theorem, !(ref).

\begin{thm}[The Open Set Condition for Continuity]
  Let $(X,d)$ and $(Y,e)$ be metric spaces and $f:X \rightarrow Y$
  a map. then f is continuous \emph{if and only if} $f^{-1}(G)$ is open
  in $(X,d)$ for all $G$ open in $(Y,e)$.
\end{thm}

This is exactly how we define what it means for a function between to
topological spaces to be continuous.

\begin{defn}
  Let $(X,\mathscr{T}_X),(Y,\mathscr{T}_Y)$ be topological spaces. We
  say that $f: X \rightarrow Y$ is a \emph{continuous function} if for
  each open $V \subseteq Y$, $f^{-1}(V)$ is open in $X$.
\end{defn}

Before we move onto the example, recall that a \emph{homeomorphism} is
a continuous function with a continuous inverse. This will be
important as we tackle the main theorem in the next section.

\begin{exmp}
  Let $X$ be a set and suppose $\mathscr{T}$ and $\mathscr{T}'$ are
  topologies on $X$. Then the identity map $id : (X,\mathscr{T})
  \rightarrow (X \mathscr{T}')$ where $id(x) = x$, is a homeomorphism
  if and only if $\mathscr{T} = \mathscr{T}'$. 
\end{exmp}

\begin{proof}
  The sufficient condition proves to be trivial;
  We have that $id$ is continuous, i.e.\ for all $V$ open under
  $\mathscr{T}'$, $id^{-1}(V) = V$ is open under $\mathscr{T}$ if and
  only if $\mathscr{T} \subseteq \mathscr{T}'$.

  Repeating the argument for $id^{-1}$, we have that $\mathscr{T}'
  \subseteq \mathscr{T}$. Combining the results we see that $id$ is a
  homeomorphism if and only if $\mathscr{T} = \mathscr{T}'$.
\end{proof}

We have the following equivalent definitions of continuity.

\begin{thm}
  Let $X,Y$ be topologies with $f : X \rightarrow Y$ a map. The
  following are equivalent.
  \begin{enumerate}
  \item $f$ is continuous.
  \item If $A \subseteq X$, then $f(\overline{A}) \subset %% closure definition
    \overline{f(a)}$.
  \item If $B \subseteq Y$ is closed then so is $f^{-1}(B)$ in $X$. %% closed definition
  \item For each $x \in X$ and each neighbourhood $V$ of $f(x)$,
    there exists $U \subseteq X$ such that $x \in U$ and $f(U)
    \subseteq V$.
  \end{enumerate}
\end{thm}

\begin{proof}
  We will show that (1) implies (2) implies (3) implies (1), and that
  (1) and (4) are equivalent.

  \begin{itemize}
  \item[(1)$\implies$(2)] Assume that $f$ is continuous, and $A
    \subset X$. If $x \in \overline{A}$ then it has each neighbourhood
    intersecting $A$. If $V$ is a neighbourhood of $f(x)$, then there
    is some such $f^{-1}(V)$. this intersects $A$ at say $y$, then
    $f(y) \in f(A)$ and $f(y) \in V$. Thus $f(x) \in
    \overline{f(A)}$. 
  \item[(2)$\implies$(3)] Assume (2). Let $B \subseteq Y$ be
    closed. We have $\overline{B} = B$. Also if $A = f^{-1}(B)$, $f(A)
    = f(f^{-1}(B)) \subseteq B$. So if $x$ is in the closure of $A$, 
    \[
      f(x) \in f(\overline{A}) \subseteq \overline{f(A)} \subseteq
      \overline{B} = B 
    \]
    Thus $x \in f^{-1}(B) = A$ and thus the closure of $A$ is
    contained in $A$.
  \item[(3)$\implies$(1)] Suppose $V$ is open in $Y$. Then $V^c$ is
    closed by definition. By (3) we then have that $f^{-1}(V^c)$ is
    closed. But
    \[
      f^{-1}(V^c) = (f^{-1}(V))^c.
    \]
    Thus the complement of $f^{-1}(V^c)$ is $f^{-1}(V)$ and
    this is open as required.
  \item[(1)$\implies$(4)] Let $f$ be continuous and suppose $x \in
    X$. If $V$ is a neighbourhood of $f(x)$ we have that  $x \in
    f^{-1}(V)$, an open set in $X$. thus we may find a basis element
    $U$ of the topology on $X$ such that $x \in U \subseteq
    f^{-1}(V)$. It follows that $f(x) \in f(U) \subseteq V$, as
    required.
  \item[(4)$\implies$(1)] Assume (4). Let $V$ be any open set in
    $Y$ and let $x \in f^{-1}(V)$. So $f(x) \in V$. By hypothesis,
    there exists some neighbourhood $U_x$ of $x$ such that $f(U_x)
    \subseteq V$. Taking the union of $U_x$ over all the $x \in V$, we
    find that it is equal to $f^{-1}(V)$ and by the second axiom, this
    is open.
  \end{itemize}  
\end{proof}

\subsection{The Quotient Topology}
\label{sec:prelims:quotient}

We now turn our attention to the quotient topology. The \emph{quotient
  space}, which we will introduce in this subsection as well, will
serve to be the structure that we reduce various surfaces to; in
general this section will start connecting the abstract set and logic
based approach we have been using to study topology and see how this
can be used in geometry.

\begin{defn}
  Given a topological spaces $X$ and  $Y$, we say that a surjective
  function $f:X \rightarrow Y$ is a \emph{quotient map} if $V
  \subseteq Y$ is open \emph{if and only if} $f^{-1}(V)$ is open in
  $X$.
\end{defn}

This condition is clearly stronger than the notion of continuity that
we introduced in section \ref{sec:prelims:continuity}. An equivalent
condition would be (with $X$, $Y$ and $f$ as before): $U \subseteq Y$
is closed \emph{if and only if} $f^{-1}(U)$ is closed in $X$. With
this in mind, we define the quotient topology as follows:

\begin{defn}
  Let $(X,\mathscr{T})$ be a topological space, $Y$ be a set and
  suppose $f: X \rightarrow Y$ is a surjective map. Then, we say that
  the \emph{quotient topology} is defined as consisting of all subsets
  of $Y$ whose preimage is open in $X$.
\end{defn}

This is indeed a topology refer to !(cite), moreover it is a topology
with respect to which $f$ is a quotient map. For a partition $X^*$ on
a topological space $X$, there is a natural surjection that maps every
element of $x \in X$ to the partition $[x]$ \footnote{Here, we write
  the partition containing $x$ as an equivalence class. The
  relation here is that which relates any two elements in the same
  partition} that contains it. With this in mind it is intuititve to
define the following.

\begin{defn}
  Given a topology $(X, \mathscr{T})$ and a partition $X^*$ of $X$, we
  define the \emph{quotient space} of $X$ to be $X^*$ under the
  quotient topology with respect to the surjection $\pi : X
  \rightarrow X^*$ given by $x \mapsto [x]$.
\end{defn}

In section \ref{sec:surf:prototype} we will introduce many examples of
quotient spaces. For now, we will move on to a theorem that will prove
useful whilst working with subspaces' quotient spaces. It is not true
in general true that restricting quotient maps to a subspace will give
us a quotient map on the subspace topology. We do have a useful result %% example needed
in the following cases however.

\begin{thm}
  Let $f:X \rightarrow Y$ be a quotient map and $A \subseteq X$ be a
  subspace that contains every set $f^{-1}(\{ y \})$ that it
  intersects \footnote{Munkres calls this property of sets with
    respect to functions \emph{saturation}. !(cite)}. If we define $g$
  to be the restriction of $f$ to $A$, then we have the following
  \begin{enumerate}
  \item If $A$ is open or closed in $X$, then $g$ is a quotient map.
  \item If $g$ is an open or closed map \footnote{We say that a map
      $f:X \rightarrow Y$ between two topologies is open or closed if
      it maps open sets to open sets or closed sets to closed sets
      respectively.}, then $g$ is a quotient map.
  \end{enumerate}
\end{thm}

\begin{proof}
  !(retvrn)
\end{proof}

\subsection{Compactness}
\label{sec:prelims:compact}

This is the final concept that we will need to introduce before we can
define surfaces and make a formal statement of the classification
theorem. We start by defining \emph{coverings}.

\begin{defn}
  Suppose $X$ is a set. We say that a collection $\mathscr{C}$ of
  subsets of $X$ \emph{covers} $X$ if the union of the elements of
  $\mathscr{C}$ is equal to $X$. When $X$ is a topological space we
  say that $\mathscr{C}$ is an \emph{open cover} of $X$ if it is a
  cover each of whose elements is open in $X$. !(equal?? or subset)
\end{defn}

With this we define compactness.

\begin{defn}
  A topological space $X$ is said to be \emph{compact} if for each
  open cover $\mathscr{C}$ of $X$, there is a finite subcollection of
  $\mathscr{C}$ that also covers $X$.
\end{defn}

Such a subcollection is also called a \emph{subcover}. Readers may be
familiar with an alternate definition of compactness concerning
sequences possessing convergent subsequences that is equivalent to the
one we have given above. We provide the following which expresses the
same idea.

\begin{thm}
  A topological space $(X, \mathscr{T})$ is compact if and only if
  each of its infinite subsets has a limit point.
\end{thm}

\begin{proof}
  !(fill or omit)
\end{proof}

Now we will take a look at topological subspaces and Hausdorff spaces
under the condition of compactness.

\begin{lem}
  Let $Y$ be a subspace of $X$. Then $Y$ is compact if and only if
  every cover of $X$ that is \emph{open in $X$} contains a finite
  subcover.
\end{lem}

\begin{proof}
  For the neccessary condition, suppose $Y$ is compact and
  $\mathscr{C}$ is a covering of $Y$ that is open in $X$. For every
  element $C \in \mathscr{C}$, we notice that $C \cap Y$ is open in
  the subspace by definition. Therefore The collection
  \[
    \mathscr{C}_Y = \{ C \cap Y \mid C \in \mathscr{C} \}
  \]
  is an open covering in the subspace and by compactness, there exists
  a finite subcover of the form $\{C_i \cap Y \mid i = 1, \dots,
  n$ for some finite $n$. Thus we have a subcollection $(C_i)_{i=1}^n$
  of $\mathscr{C}$ that covers $Y$ as required.

  Conversely suppose every covering of $Y$ that is open of $X$ has a
  finite subcover. We observe that every element of $C_\alpha'$ of an
  arbitrary open cover $\mathscr{C}'$ in the subspace $Y$ can be
  written as the intersection of $Y$ with an element $C_\alpha$ open
  in $X$. Furthermore the set $\{C_\alpha \}$ is a cover of $Y$ open
  in $X$ and by assumption has a finite subcollection $\{C_{\alpha_1},
  C_{\alpha_2}, \dots, C_{\alpha_n} \}$ that covers $Y$. Clearly we
  then have that the subcollection of $\mathscr{C}'$,
  $\{C_{\alpha_1}', C_{\alpha_2}', \dots, C_{\alpha_n}' \}$ covers
  $Y$. 
\end{proof}


%%% Local Variables:
%%% mode: latex
%%% TeX-master: "main"
%%% End:
